\section{Introduction}
The project has progressed and we now have prototypes of the core components completed as well as a containing framework structure (the dragonfly-brain class) for the entire model, which each component can be integrated into.

To ensure efficiency, utility and functionality of our code we have conducted unit and coverage tests for the functions of each component. Alongside ensuring functionality with unit tests, we have also conducted time profiling upon our code to determine how performance bottlenecks that would inhibit near real time execution could be removed or improved. Although functionality has been achieved for the core modules (see Figure \ref{fig:structure}), these tests revealed performance may inhibit real time execution.

In the process of developing these components we have come across several challenges that we have had to respond to. In addition, certain tasks included in our original specification no longer appear to be either feasible because of encountered limitations, or are no longer relevant as they have been found to be unnecessary to achieve the core goals of the project. Because of our use of the Agile development method, we have been able to adapt our development strategy to respond to these issues, and have recomposed our schedule and specification as a result.

We will first discuss progress made for each individual component, followed by the revised task schedule and finally the testing methodology for each component accompanied by test results.
