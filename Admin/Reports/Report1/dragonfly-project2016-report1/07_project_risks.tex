\section{Project Risks} \label{projectrisks}

\par  There may be significant overhead required for the team to become fully acquainted with the neurodynamic concepts needed. As such we are maintaining open channels of communication with our supervisors and have completed background reading following an initial project meeting on 16$^{th}$ December 2016.

\par Associated with this overhead is the need (see requirement B3), not only to produce functioning code, but also a model that exhibits the same characteristics as those found in published experiments. In order to mitigate this we are developing the CSTMD1 iteratively and at a high priority to ensure that it is accurate so other modules can be developed from it. A visualiser has also been developed allowing the team to observe the voltage changes through the dendritic tree.

\par Because of the complexity of each module, it will be advantegous to have a group member with experience of a module working on it at all times. However, last year's group encountered the problem of excessive independant specialisation\cite{GITHUB2}, and when problems arose with a specific module other group members were too inexperienced with the module to aid in its development. To mitigate this risk whilst also retaining experience, group members will alternatively cycle between modules every two sprints, keeping an experienced developer on a module whilst ensuring all members gain experience over all modules.

\iffalse
\par Due to project aim to create a pseudo-realtime system there is an ongoing risk that one or more modules cannot operate at the required sample time. Given the feedback from the previous MSc project \cite{GITHUB2} it was seen that the CSTMD1 simulation acted as a bottleneck resulting in an end product that required manual batch data manipulation. During initial work optimisations have already been found in order to minimise the risk associated with computation time including but not limited to removing inefficiecies in the inter-module interactions.
\fi