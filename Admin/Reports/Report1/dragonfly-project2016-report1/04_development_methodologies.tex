\section{Development Methodology} \label{developmentMethodology}

 Since our project has complex, interdependent modules that require implementation in different phases, and a large number of possible extensions, we have chosen to utilise an iterative Agile development practice with Scrum methods.

 The project has been split into smaller goals which have been priotised into sprints (see Section \ref{taskscheduling}). We also have decided to apply the practice of pair programming for these sprints. Tasks from a sprint will be assigned to a pair who will work together and review each other's code. At the end of each sprint, one member of the pair shall be cycled to work on a different module. This mitigates the risk last year's group encountered of individual specialisation (see Section \ref{projectrisks}) by ensuring each group member gains experience of each of the modules, whilst simultaneously ensuring a developer with experience of the module is involved with it.

We are currently using Slack, a communication platform for internal project communiation. Each module has a dedicated slack channel allowing for efficient communication between team members working on the same functionality. Task management and assignment will be handled using online task collaboration tool Trello, which will manage the tasks involved for the goals of each sprint. We decided to use Git as our version control system with our remote repository hosted on the Imperial Department of Computing GitLab. 